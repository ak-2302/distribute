\documentclass[12pt,a4paper,dvipdf]{jsarticle}
\usepackage{listings}
\usepackage{amsmath, xparse}
\usepackage{xcolor}
\usepackage[dvipdfmx]{graphicx}
\usepackage{float}
\usepackage{url}
\setlength{\textwidth}{\textwidth}
\setlength{\topmargin}{0pt}
\setlength{\headheight}{0pt}
\setlength{\headsep}{0pt}
\lstset{
    basicstyle = {\ttfamily}, % 基本的なフォントスタイル
    frame = {tbrl}, % 枠線の枠線。t: top, b: bottom, r: right, l: left
    breaklines = true, % 長い行の改行
    numbers = left, % 行番号の表示。left, right, none
    showspaces = false, % スペースの表示
    showstringspaces = false, % 文字列中のスペースの表示
    showtabs = false, % タブの表示
    keywordstyle = \color{blue}, % キーワードのスタイル。intやwhileなど
    commentstyle = {\color[HTML]{1AB91A}}, % コメントのスタイル
    identifierstyle = \color{black}, % 識別子のスタイル 関数名や変数名
    stringstyle = \color{brown}, % 文字列のスタイル
    captionpos = t % キャプションの位置 t: 上、b: 下
}
\title{簡易プロトタイピングによる\\ユーザインタフェース設計\\検討会説明用資料}
\author{202210779 山田 悠真\\202211879 新井 皓陽\\202212115 近 和}
\date{\today}
\begin{document}
\maketitle
\newpage
% \section{sample}
% \begin{figure}[H]
%     \centering
%     \includegraphics[width=10cm]{sample.png}
%     \caption{sample.png}
%     \label{fig:sample}
% \end{figure}


\section{制作するユーザーインターフェースの概要}
\subsection{制作するシステムを選定した経緯}
この実験では「空き教室確認システム」と題して教室の空き状況と設備情報を確認するシステムを作成した。対象システム制定の経緯としては、現在存在する学内マップや授業管理システムでは教室に着目したユーザーインターフェースがなく、教室ごとの利用状況を把握することが困難であるという問題があった。そのため、空き教室を確認するためのユーザーインターフェースを作成することを目標とした。また、キャンパスマップには学内マップに教室ごとの設備情報が掲載されているが、表形式で纏められているため教室の位置を把握することが困難である。故に、教室の設備情報を教室の位置情報と関連付けたユーザーインターフェースを作成することを目標とした。
\subsection{実験参加者に課す課題内容}
実験を行っていく間に発生した問題点を踏まえて逐次実験の手順を改善したため、最初の実験参加者に行った指示と異なる部分がある。
\subsubsection{「現在自分がいる建物内にある暗幕のある空き教室を見つけ、その教室番号を答える」}
この課題では、実験参加者は予め説明を受けた現在地マーカーを頼りに現在の自分が居る建物を確認する。その後、実験参加者は建物内マップのアイコンを頼りに階層を移動し、暗幕がある空き教室を把握することを想定している。また、この課題では、馴染みのあるエリアと馴染みのないエリアでの操作の差異を確認するため、現在地を3C棟に配置した場合と6A棟に配置した場合をシミュレーションした。

\subsubsection{「以下の条件をすべて満たす教室の教室番号を答える」}
この課題では、実験参加者は絞り込みの機能を用いて条件に適合する教室を把握し、その教室番号を答えることを想定している。\\
課す条件は以下の通りである。
\begin{itemize}
    \item 液晶プロジェクターが利用可能
    \item 3C棟内にある
    \item 1月31日 金曜日の15時15分から18時に利用可能である
    \item 収容人数が20人以上である
\end{itemize}

\subsubsection{「検索窓を用いて6A202ビジュアルデザイン室および3C113計算機室の利用予定表を確認する」}
この課題では、実験参加者は検索窓を用いて対象教室のの利用予定表を確認することを想定している。そのため、実験参加者には必ず検索窓を用いる旨を周知する。また、課題の対象となる教室については課題1と同様の理由で複数の教室を挙げているが、実験時間の都合で実験参加者にはそれぞれ片方の教室のみについて行ってもらった。
\subsubsection{実験参加者に事前に周知する点}
\begin{itemize}
    \item 作成したシステムは筑波大学の空き教室を調べるサービスのシステムである
    \item 赤色に縁取られた教室は空き教室である
    \item 画面上部のアイコンから「エリア選択」、「キーワード検索」、「絞り込み検索」が行える
    \item 付箋をタップで文字を入力できる
    \item 課題は1から順に行う
    \item 各課題の開始時には大学全域地図の画面に初期化する
    \item 全課題終了後にアンケートがある
    \item 不明な点があれば課題開始前に口頭で質問を受け付ける
\end{itemize}

\newpage
\section{作成する画面のリストアップ}
\begin{enumerate}
    \item 全学エリアの地図
    \item 第123エリアの地図
    \item 第3エリアの地図
    \item 大学会館・体育・芸術エリアの地図
    \item 3C棟1階の地図
    \item 3C棟2階の地図
    \item 3C棟3階の地図
    \item 3C棟4階の地図
    \item 6A棟1階の地図
    \item 6A棟2階の地図
    \item 6A棟3階の地図
    \item 6A棟4階の地図
    \item 3C113の設備情報と利用予定表
    \item 6A202の設備情報と利用予定表
    \item 検索窓
    \item 検索画面
    \item 絞り込みタブ
    \item エリア選択タブ
    \item 絞り込み結果
\end{enumerate}

\newpage
\section{作成する画面のデザイン}
\begin{enumerate}
    \item 全学エリアの地図
    \item エリアの地図
    \item 棟の地図
    \item 教室の情報と利用予定表
    \item 検索
    \item エリア選択
    \item 絞り込み
    \item 絞り込み中の地図
\end{enumerate}
\begin{figure}[H]
    \centering
    \begin{minipage}[b]{0.24\columnwidth}
        \centering
        \includegraphics[width=0.9\columnwidth]{./img/全学地図.png}
        \caption{全学地図}
        \label{fig:全学地図}
    \end{minipage}
    \begin{minipage}[b]{0.24\columnwidth}
        \centering
        \includegraphics[width=0.9\columnwidth]{./img/エリア地図.png}
        \caption{エリア地図}
        \label{fig:エリア地図}
    \end{minipage}
    \begin{minipage}[b]{0.24\columnwidth}
        \centering
        \includegraphics[width=0.9\columnwidth]{./img/棟地図.png}
        \caption{棟地図}
        \label{fig:棟地図}
    \end{minipage}
    \begin{minipage}[b]{0.24\columnwidth}
        \centering
        \includegraphics[width=0.9\columnwidth]{./img/検索.png}
        \caption{検索}
    \end{minipage}
\end{figure}
\begin{figure}[H]
    \centering
    \begin{minipage}[b]{0.24\columnwidth}
        \centering
        \includegraphics[width=0.9\columnwidth]{./img/エリア選択.png}
        \caption{エリア選択}
    \end{minipage}
    \begin{minipage}[b]{0.24\columnwidth}
        \centering
        \includegraphics[width=0.9\columnwidth]{./img/絞り込み.png}
        \caption{絞り込み}
    \end{minipage}
    \begin{minipage}[b]{0.24\columnwidth}
        \centering
        \includegraphics[width=0.9\columnwidth]{./img/絞り込み中.png}
        \caption{絞り込み中}
    \end{minipage}
    \begin{minipage}[b]{0.24\columnwidth}
        \centering
        \includegraphics[width=0.9\columnwidth]{./img/部屋情報.png}
        \caption{部屋情報}
    \end{minipage}
\end{figure}



\newpage
\section{ユーザビリティテスト}
\subsection{取得データと取得方法}
この実験での取得データを決定するにあたり、操作に要した時間を計測することの有用性について考証した。KLMに基づいたボタンごとの操作難易度と想定操作時間を計算すると、ボタンごとの想定操作時間に大きな差がみられなかった。そのため、本実験の少ないサンプル数では操作に要した時間の計測は取得データとして不適切であると考え、棄却した。\\
ゆえに、この実験では取得データとして操作のタップ回数を計測した。実験時に実験参加者の操作を動画に記録し、通常のタップ数と文字入力の回数を動画から測定した。加えて、キャンパスマップを用いて課題1を行ったデータを我々を被験者として計測した。なお、このキャンパスマップで課題1を行ったデータは、実際には得られる情報が課題1より少なく、そのため、キャンパスマップを用いたほうが有利であるといえる。\\
また、実験参加者には課題終了後にアンケートに回答してもらい、そのアンケートフォームに寄せられた回答から課題点を読み取った。
\subsection{データの解析方法}
アンケートによって取得したデータは以下の図のように根本となる原因を深堀することで問題点を顕現させた。
\begin{figure}[H]
    \centering
    \includegraphics[width=0.8\columnwidth]{./img/アンケート分析.png}
    \caption{アンケート分析}
\end{figure}
参考文献から、各操作にかかる秒数を求めることができるため、その操作にかかる時間を操作に対する負荷と考える。\\
\begin{figure}[H]
    \centering
    \begin{tabular}{c|c|c|c}
        操作名      & タップ     & ズーム     & フリック    \\
        \hline
        操作にかかる秒数 & 0.20241 & 0.20927 & 0.11406
    \end{tabular}\\
\end{figure}
作成したプロトタイプとキャンパスマップについて、実験で取得したデータによる負荷スコアを計算する。負荷スコアは以下の式によって定義する。
$$
    \text{負荷スコア} = \cfrac{1}{\text{実験参加者数}}\sum_{\{\text{実験参加者}\}} \sum_{\{\text{タップ}, \text{ズーム}, \text{フリック} \}} \text{各操作にかかる秒数} \times \text{各操作の回数}
$$
ここで、負荷スコアが小さいほど操作に負荷が小さく、良いシステムであると言える。

\subsection{データ解析の結果}
\subsubsection{課題1}
課題1-1で取得したデータ\\
\begin{figure}[H]
    \centering
    \begin{tabular}{c|c|c|c|c|c|c|c|c}
               & 理論値 & 1人目 & 2人目 & 3人目 & 4人目 & 5人目 & 6人目 & 平均値         \\
        \hline
        合計タップ数 & 6   & 6   & 6   & 6   & 6   & 16  & 12  & 8.666666667
    \end{tabular}\\
\end{figure}
課題1-2で取得したデータ\\
\begin{figure}[H]
    \centering
    \begin{tabular}{c|c|c|c|c|c|c|c|c}
               & 理論値 & 1人目 & 2人目 & 3人目 & 4人目 & 5人目 & 6人目 & 平均値         \\
        \hline
        合計タップ数 & 5   & 5   & 6   & 6   & 5   & 5   & 5   & 5.333333333
    \end{tabular}\\
\end{figure}
上記の計算式によって算出した負荷スコアは以下のようになった。
\begin{figure}[H]
    \centering
    \begin{tabular}{c|cc}
              & 作成したプロトタイプ & キャンパスマップ \\
        \hline
        負荷スコア & 1.8        & 2.0
    \end{tabular}\\
\end{figure}
ここで、キャンパスマップの実験では「特定の設備を持つ教室名」の情報を得られるのに対し、プロトタイプの実験では「特定の設備を持つ教室名」、「特定の設備を持つ教室の場所」、「教室の空き状況」、「現在地に対しての教室の位置」という情報を得ることができる。これを加味すると、キャンパスマップよりも今回作成したプロトタイプが優位であることは自明といえよう。
\subsubsection{課題2}
課題2で取得したデータ\\
\begin{figure}[H]
    \centering
    \begin{tabular}{c|c|c|c|c|c|c|c|c}
               & 理論値 & 1人目 & 2人目 & 3人目 & 4人目 & 5人目 & 6人目 & 平均値         \\
        \hline
        合計タップ数 & 7   & 17  & 17  & 7   & 6   & 20  & 9   & 12.66666667
    \end{tabular}\\
\end{figure}
課題2については、多くの実験参加者において課題を正常に行うに至らず、十分なデータ数が取れなかった。そのため、実験の操作についての負荷スコアを算出することが不適切であると考えられる。ゆえに、課題2についてはアンケートの結果のみについて分析を行った。
\subsubsection{課題3}
課題3で取得したデータ\\
\begin{figure}[H]
    \centering
    \begin{tabular}{c|c|c|c|c|c|c|c|c}
               & 理論値 & 1人目 & 2人目 & 3人目 & 4人目 & 5人目 & 6人目 & 平均値         \\
        \hline
        合計タップ数 & 3   & 3   & 4   & 3   & 3   & 3   & 3   & 3.166666667
    \end{tabular}\\
\end{figure}
課題3で行った検索操作については実験参加者の中でデータの分布が非常に小さくなった。これは、検索のUIを一般的なものと同様にしたために新規システムを触るのではなく、既存のシステムを触る感覚で直感的に操作できたものと考えられる。よって課題3についての具体的な改善点は見受けられなかった。
\subsubsection{想定したユーザー行動との相違}
\noindent
インターフェースについての想定したユーザー行動との相違
\begin{itemize}
    \item 課題1において、暗幕のみで絞り込み機能を使った
    \item スクリーンと液晶プロジェクタの選択間違い
    \item 空いていない教室を選択した
    \item 現在地が3Cにあるときに3Aを選択した
    \item 現在地が3Cにあるときに3Dを選択した
    \item 液晶プロジェクタと暗幕の選択間違い
    \item 暗幕のない空き教室を選択した
\end{itemize}
ユーザビリティテストの進め方についての想定したユーザー行動との相違
\begin{itemize}
    \item 3学を指示したが2学を選択した
    \item 教室名を言う課題で教室名を言わずに教室をタップした
\end{itemize}

\subsubsection{アンケートに記入された意見}
\begin{itemize}
    \item 利用予定表のメモリが少なく、時間がわからない
    \item 現在地が曖昧に感じた
    \item ハンバーガーアイコンがハンバーガーに見えない
    \item フィルター機能に気づくのに時間がかかる
    \item 課題3の終了のタイミングがわからなかった
    \item 課題1で近くの建物も検索対象かと思った。
    \item 現在地の詳細な地図に一度で飛べる手段があるとよい
\end{itemize}
\subsubsection{問題点}
インターフェースについての問題点
\begin{itemize}
    \item 現在の空き状況を考える必要がある課題で現在の空き状況を考慮しない絞り込みが行われた
    \item 絞り込みのアイコンの押し間違い
    \item 現在地がない場所にある教室を選択した
\end{itemize}
\subsubsection{問題点の分析とプロトタイプの修正}
\begin{enumerate}
    \item 利用予定表のメモリが少なく、時間がわからない\\
          ⇒ 利用予定表を授業時間を基準とした1限から6限までの6つの枠に変更した。授業の開始時間と終了時間を記載した。
    \item 現在地が曖昧に感じた\\
          ⇒ 地図に重ねた透明なフィルムが撓んでしまっていたため、その上に配置した現在地マーカーとマップの地点の対応が認識しにくかったと考えられる。また、現在地マーカーとして用いたアイコンが一般に目的地や選択位置に使われるアイコンであったため、現在地という認識が困難であったと推測できる。この問題点を解決するには、現在地マーカーのデザインを一般に起用されるものに変更するとよい。
    \item ハンバーガーアイコンがハンバーガーに見えない\\
          ⇒ 一般に通用している呼び方であるため、実験に問題はない。
    \item フィルター機能に気づくのに時間がかかる\\
          ⇒ 事前に絞り込み機能があることに加えて、それを起動するボタンの位置を事前に説明しているため、実験に問題はない。
    \item 課題3の終了のタイミングがわからなかった\\
          ⇒ 課題の終了条件を「空いているか確認する」としていたが、「空いているか確認して報告する」に変更することで終了条件を明確にした。終了条件の変更後に課題3の終了条件に悩む実験参加者は確認されなかった。
    \item 課題1で近くの建物も検索対象かと思った\\
          ⇒ 事前に課題内容で「現在地を示すので、その建物内の(略)」と指示を行っている。この指示は十分一意に定まると考えられるため、実験に問題はない。
    \item 現在地の詳細な地図に一度で飛べる手段があるとよい\\
          ⇒ 今回の課題では現在地への遷移はたかだか3回の操作で可能である。本実験ではペーパープロトタイピングの都合上、画面の遷移に時間を要したが、動作速度が十分速い本番環境を想定すると「現在地の詳細な地図に一度で飛べる手段」の重要性はそれほど高くないと考えられる。この機能を導入することによる、マップ表示に利用可能な画面領域の減少とユーザーの利益を慎重に比較する必要があるといえる。
    \item 現在の空き状況を考える必要がある課題で現在の空き状況を考慮しない絞り込みが行われた\\
          ⇒ 今回実装した絞り込み機能では、時間指定の機能を付けたため、時間指定をしない場合には全教室の中から絞り込み対象を検索する仕様となっていた。そのため、課題1で暗幕のみの絞り込みを行った場合には、課題に示した「今空いている」という条件を無視した絞り込み結果が出てしまい、その教室が現在空いているかの精査をしない実験参加者が複数人見られた。これを解決するには時間指定のされていない場合には現在の空き情報を絞り込み候補一覧に追加することで解決できると考えられる。
    \item 現在地がない場所にある教室を選択した\\
          ⇒ ダミーデータを画面に出すことで対策した。
\end{enumerate}
\section{参考文献}
\begin{enumerate}
    \item El Batran, K., \& Dunlop, M. D. (2014). Enhancing KLM (Keystroke-Level Model) to fit touch screen mobile devices. Proceedings of the 16th International Conference on Mobile Human-Computer Interaction (MobileHCI '14), 23-26 September 2014, Toronto, ON, Canada (pp. 2628363-5). ACM. \url{https://doi.org/10.1145/2628363.2628385}
\end{enumerate}

\end{document}