\documentclass[12pt,a4paper,dvipdf]{jsarticle}
\usepackage{listings}
\usepackage{amsmath, xparse}
\usepackage{xcolor}
\usepackage[dvipdfmx]{graphicx}
\usepackage{float}
\setlength{\textwidth}{\textwidth}
\setlength{\topmargin}{0pt}
\setlength{\headheight}{0pt}
\setlength{\headsep}{0pt}
\lstset{
    basicstyle = {\ttfamily}, % 基本的なフォントスタイル
    frame = {tbrl}, % 枠線の枠線。t: top, b: bottom, r: right, l: left
    breaklines = true, % 長い行の改行
    numbers = left, % 行番号の表示。left, right, none
    showspaces = false, % スペースの表示
    showstringspaces = false, % 文字列中のスペースの表示
    showtabs = false, % タブの表示
    keywordstyle = \color{blue}, % キーワードのスタイル。intやwhileなど
    commentstyle = {\color[HTML]{1AB91A}}, % コメントのスタイル
    identifierstyle = \color{black}, % 識別子のスタイル 関数名や変数名
    stringstyle = \color{brown}, % 文字列のスタイル
    captionpos = t % キャプションの位置 t: 上、b: 下
}
\title{簡易プロトタイピングによる\\ユーザインタフェース設計\\ユーザビリティテスト3回目レポート}
\author{202210779 山田 悠真\\202211879 新井 皓陽\\202212115 近 和}
\date{\today}
\begin{document}
\maketitle
\newpage
% \section{sample}
% \begin{figure}[H]
%     \centering
%     \includegraphics[width=10cm]{sample.png}
%     \caption{sample.png}
%     \label{fig:sample}
% \end{figure}
\section{明らかになった問題点とそれに対する改善}
\subsection{インターフェースについての問題点とアンケート結果}
\begin{itemize}
      \item 利用予定表の時間配分が離散的でわかりづらい\\
            → 授業時間に即したモードとそれ以外を切り替えることができる機能を追加すると解決できる。
      \item 絞り込み機能に気づきにくい\\
            → あらかじめ絞り込み画面の起動方法を説明しているため、不問。
      \item 初期画面から現在地がある場所に飛ぶ手段があったほうが良い\\
            → 3回のタップで遷移できるようになっているため、動作速度が十分な場合を考えると必ずしも必要とは言えないが、機能を追加する意義はある。
      \item 検索機能が提示する選択肢が現在のマップ内にあるもののみにするとよい\\
            → 制作時に協議を重ね、現在のマップ外の候補も表示したほうが役に立つ場合が多いと考え、今の仕様とした。現在のマップ内のものを優先的に提示するようにはなっている。
      \item 絞り込みやアイコンに教室の椅子の種類の情報があるとよい\\
            → 参考にしたキャンパスマップにある主要なアイコンは採用したために需要の多くを満たしていると考えていたが、一般的に大型機材の所在については絞り込み機能で十分に需要を満たせそうであるため、「最寄りの~がある部屋を知りたい」という気軽なニーズに応える情報をアイコンとして優先すべきであった。
\end{itemize}
\subsection{プロトコルについての問題点とアンケート結果}
\begin{itemize}
      \item 課題1で近くの建物も検索対象かと思った\\
            → 現在自分がいる建物内と課題で明示しているため、不問。
      \item 「ハンバーガーアイコン」がハンバーガーに見えない\\
            → 指さしでの説明を行っていたため、不問。
      \item 現在地が曖昧に感じた。\\
            → フィルムが立体的に曲がってしまっていたため、その上に置いた現在地マーカーが3次元的にずれてしまったと考えられる。フィルムの下に現在地マーカーを差し込むべきであった。
\end{itemize}
\end{document}